\documentclass[a4paper,12pt]{article}
\usepackage[utf8]{inputenc}
\usepackage[T1]{fontenc}
\usepackage[brazil]{babel}
\usepackage{geometry}
\usepackage{fancyhdr}
\usepackage{xcolor}
\usepackage{tabularx}
\usepackage{lastpage}
\usepackage{hyperref}
\usepackage{colortbl}
\usepackage{enumitem}
\usepackage{textcomp} % Suporte para símbolos de texto
\usepackage{amssymb}  % Suporte para símbolos matemáticos
\DeclareUnicodeCharacter{2264}{\ensuremath{\le}}         % Corrige o erro 2264 (≤)
\DeclareUnicodeCharacter{2082}{\textsubscript{2}}        % Corrige o erro 2082 (₂ de CO₂)
\DeclareUnicodeCharacter{2265}{\ensuremath{\ge}}         % Código 2265: Maior ou igual (≥)
\DeclareUnicodeCharacter{2248}{\ensuremath{\approx}}     % Código 2248: Aproximadamente (≈)

% Configuração das Margens
\geometry{left=2.5cm, right=2.5cm, top=3cm, bottom=3cm}

% Configuração do Cabeçalho e Rodapé
\pagestyle{fancy}
\fancyhf{}
\renewcommand{\headrulewidth}{1pt}
\lhead{Sistema SAFE\\Plano de Testes} 
\rhead{\vspace{0.4cm}Versão 1.0.0}

% Rodapé conforme o original
\fancyfoot{} 

% Define um único campo central contendo uma tabela que organiza os 3 elementos
\cfoot{
    \scriptsize % Reduz levemente a fonte para garantir que caiba na linha
    \begin{tabular*}{\textwidth}{@{\extracolsep{\fill}} l c r @{}}
        http://www.cos.ufrj.br/$\sim$ese & 
        \textbf{LENS} – Laboratório de Engenharia de Software & 
        Página \thepage\ de \pageref{LastPage}
    \end{tabular*}
}

\begin{document}

% --- CAPA ---
\begin{titlepage}
    \thispagestyle{empty}
    \vspace*{6cm}
    \begin{flushright}
        {\Huge \textbf{Sistema SAFE}}\\
        \vspace{0.5cm}
        {\Huge \textbf{Plano de Testes}}\\
        \vspace{1cm}
        {\Large \textbf{Versão 1.0.0}}
    \end{flushright}
    
    \vspace{2cm}
    \noindent
    \textbf{Organização:} LENS – Laboratório de Engenharia de Software\\
    \textbf{Sistema:} SAFE (Sistema de Apoio à Biossegurança em Ambientes Físicos)
\end{titlepage}

% --- HISTÓRICO ---
\newpage
\centerline{\Large \textbf{Histórico de Revisão}}
\vspace{0.5cm}

\begin{tabularx}{\textwidth}{|l|l|X|l|}
\hline
\rowcolor{gray!30} \textbf{Data} & \textbf{Versão} & \textbf{Descrição} & \textbf{Autor} \\ \hline
2025-12-15 & 1.0.0 & Criação inicial do plano de testes & Equipe QA \\ \hline
\end{tabularx}

\vspace{1cm}

% --- SUMÁRIO ---
\tableofcontents
\newpage

% --- CONTEÚDO ---

\section{Introdução}

\subsection{Finalidade}

Este documento tem como finalidade estabelecer a estratégia de testes para o Sistema SAFE, um sistema IoT de missão crítica para monitoramento de biossegurança em ambientes físicos. O plano define os tipos de testes, recursos necessários, cronograma e casos de teste detalhados para execução por equipe externa e independente.

\subsection{Escopo}

O documento cobre o planejamento de testes para os três subsistemas do SAFE:

\begin{itemize}
    \item \textbf{SAFE BIoT:} Hardware/Firmware de coleta de dados ambientais
    \item \textbf{SAFE Manager:} Backend/API de gerenciamento via MQTT
    \item \textbf{SAFE Dashboard:} Frontend de visualização e alertas
\end{itemize}

Este plano está associado ao projeto SAFE e afeta diretamente:
\begin{itemize}
    \item Validação dos requisitos funcionais e não funcionais
    \item Garantia de qualidade para ambientes de produção
    \item Certificação de segurança e confiabilidade do sistema IoT
\end{itemize}

\subsection{Definições, Acrônimos, e Abreviações}

\begin{tabularx}{\textwidth}{|l|X|}
\hline
\rowcolor{gray!20} \textbf{Termo} & \textbf{Descrição} \\ \hline
SAFE & Sistema de Apoio à Biossegurança em Ambientes Físicos \\ \hline
BIoT & Dispositivo IoT composto por sensores de temperatura, movimento, CO₂ e VOCs \\ \hline
MQTT & Message Queuing Telemetry Transport - Protocolo de mensagens para IoT \\ \hline
Broker & Agente intermediário (RabbitMQ) usando protocolo MQTT \\ \hline
CO₂ & Dióxido de carbono, medido em ppm (partes por milhão) \\ \hline
VOCs & Compostos orgânicos voláteis, medidos em ppb (partes por bilhão) \\ \hline
RF & Requisito Funcional \\ \hline
RNF & Requisito Não Funcional \\ \hline
E2E & End-to-End (Teste de ponta a ponta) \\ \hline
API & Application Programming Interface \\ \hline
JSON & JavaScript Object Notation \\ \hline
Speck & Algoritmo de criptografia leve usado no sistema \\ \hline
\end{tabularx}

\vspace{0.5cm}
Para definições completas, consultar documento \texttt{SAFE\_Requisitos\_V2.pdf} (Seção "Informações e Variáveis").

\subsection{Referências}

\begin{tabularx}{\textwidth}{|X|l|l|X|}
\hline
\rowcolor{gray!20} \textbf{Título} & \textbf{Versão} & \textbf{Data} & \textbf{Onde pode ser obtido} \\ \hline
Requisitos do Sistema SAFE & 1.0 & 2025-12 & SAFE\_Requisitos\_V2.pdf \\ \hline
Informações e Variáveis SAFE & 1.0 & 2025-12 & SAFE\_Requisitos\_V2.pdf \\ \hline
Template Plano de Teste LENS & 1.0.0 & - & Template\_para\_Plano\_de\_Testes.pdf \\ \hline
Template Casos de Teste LENS & 1.0 & - & Template\_CT\_Portugues.pdf \\ \hline
\end{tabularx}

\subsection{Visão geral}

Este documento está organizado em três seções principais:

\begin{itemize}
    \item \textbf{Seção 2 - Estratégia de Teste:} Define os tipos de testes a serem realizados, cronograma, recursos e casos de teste para cada categoria
    \item \textbf{Seção 3 - Resultados dos Testes:} Área reservada para registro de resultados após execução
\end{itemize}

\newpage
\section{Estratégia de Teste}

\subsection{Abordagem Geral}

A estratégia foi desenvolvida para \textbf{maximizar a cobertura de requisitos com esforço mínimo}, seguindo os seguintes princípios:

\subsubsection{Princípios de Otimização}

\begin{enumerate}
    \item \textbf{Agrupamento por Fluxos E2E:} Testes que cobrem múltiplos requisitos em um único fluxo de dados (Sensor → Broker → Manager → Dashboard)
    \item \textbf{Priorização de Criticidade:} Foco em funcionalidades críticas de missão (monitoramento, alertas, segurança)
    \item \textbf{Testes de Limites (Boundary):} Utilização das constantes definidas como oráculos de teste
    \item \textbf{Simulação Realista:} Uso de mensagens MQTT reais e simuladores de hardware quando necessário
\end{enumerate}

\subsubsection{Fluxos Críticos Identificados}

Os seguintes fluxos E2E cobrem a maior parte dos requisitos do sistema:

\paragraph{Fluxo 1: Ciclo Completo de Dados de Qualidade do Ar}
\textbf{Cobertura:} BIoT RF03, RF04, RF06, RF07, RNF2, RNF10 | Manager RF13, RNF01, RNF03, RNF07, RNF16, RNF19 | Dashboard RF01, RF02, RF09, RF10, RNF01, RNF03, RNF05

\textbf{Descrição:} Coleta de dados ambientais (temp, CO₂, umidade, VOCs) → Publicação MQTT → Recepção Manager → Armazenamento BD → Disponibilização API → Exibição Dashboard

\paragraph{Fluxo 2: Contagem de Pessoas e Controle de Ocupação}
\textbf{Cobertura:} BIoT RF02, RNF2, RNF11 | Manager RF13, RF15, RF16, RNF04 | Dashboard RF03, RF05

\textbf{Descrição:} Detecção entrada/saída → Atualização contador → Verificação limites → Mudança de estado instalação → Notificação usuários

\paragraph{Fluxo 3: Gerenciamento de Solicitações de Manutenção}
\textbf{Cobertura:} Manager RF04, RF05, RF06, RF07, RF14

\textbf{Descrição:} Solicitação serviço → Notificação equipe → Aceite → Conclusão → Histórico

\paragraph{Fluxo 4: Gerenciamento de Solicitações de Limpeza}
\textbf{Cobertura:} Manager RF08, RF09, RF10, RF11, RF14

\textbf{Descrição:} Similar ao Fluxo 3, mas para serviços de limpeza

\paragraph{Fluxo 5: Alertas e Limites de Biossegurança}
\textbf{Cobertura:} BIoT RF01, RF05, RNF3, RNF4, RNF12 | Manager RNF02, RNF35 | Dashboard RF04, RF11, RF12

\textbf{Descrição:} Configuração limites Manager → Publicação Broker → Recepção BIoT → Monitoramento contínuo → Alerta visual/sonoro

\newpage
\subsection{Testes Funcionais}

\textbf{Objetivo:} Validar se as funcionalidades do sistema atendem aos requisitos funcionais especificados, testando os fluxos de negócio e casos de uso principais.

\subsubsection{Testes End-to-End (E2E)}

\paragraph{Prazo para realização}

\textbf{Momento:} Após integração completa dos três subsistemas\\
\textbf{Duração estimada:} 3 dias\\
\textbf{Frequência:} A cada versão release do sistema

\paragraph{Recursos necessários}

\textit{Hardware:}
\begin{itemize}[noitemsep]
    \item 1 dispositivo BIoT real (ou simulador MQTT configurado)
    \item Ambiente de rede local (Wi-Fi isolado para testes)
    \item Computador para execução do Manager
    \item Computador/dispositivo móvel para acesso ao Dashboard
\end{itemize}

\textit{Software:}
\begin{itemize}[noitemsep]
    \item Broker MQTT (RabbitMQ ou Mosquitto)
    \item Simulador de mensagens MQTT (MQTT.fx, MQTT Explorer ou script Python)
    \item Navegadores: Chrome 118+, Firefox 115+, Edge 118+, Safari 16+
    \item Banco de dados de teste (configurado)
    \item Docker (para deploy Manager/Dashboard)
\end{itemize}

\textit{Humanos:}
\begin{itemize}[noitemsep]
    \item 2 testadores (um para executar, um para validar)
    \item 1 desenvolvedor de suporte (para dúvidas técnicas)
\end{itemize}

\textit{Ferramentas de Simulação Requeridas:}
\begin{itemize}[noitemsep]
    \item Simulador de Sensor: Script Python que publica JSON no broker simulando sensores
    \item Gerador de Carga: Para simular múltiplos dispositivos BIoT simultaneamente
    \item Mock de Autenticação: Para simular diferentes papéis de usuário
\end{itemize}

\paragraph{Requisitos Funcionais a serem testados}

\textbf{Subsistema BIoT:} RF01, RF02, RF03, RF04, RF05, RF06, RF07\\
\textbf{Subsistema Manager:} RF01--RF18\\
\textbf{Subsistema Dashboard:} RF01--RF12

\paragraph{Casos de Teste}

Referência: Documento \textbf{``Casos de Teste SAFE.pdf''} para scripts detalhados.

\textbf{CT-E2E-01: Ciclo Completo de Dados de Qualidade do Ar}

\textit{Cenário:} Validar fluxo completo de coleta → transmissão → processamento → exibição de dados ambientais

\textit{Dados de Entrada:} JSON MQTT com temperatura=23.5°C, umidade=55\%, CO₂=650ppm, VOCs=120ppb

\textit{Resultado Esperado:} (1) Dados armazenados no BD do Manager, (2) Dashboard exibe valores corretos em ≤15s, (3) Dados criptografados no trânsito (Speck)

\vspace{0.3cm}
\textbf{CT-E2E-02: Contagem de Pessoas e Mudança de Estado}

\textit{Cenário:} Simular 16 entradas em instalação com limite de 15 pessoas

\textit{Dados de Entrada:} 16 eventos de entrada (entry\_flow=1) via MQTT

\textit{Resultado Esperado:} (1) Contador atualizado para 16 pessoas, (2) Estado da instalação muda para ``Não Liberado (Em uso)'', (3) Dashboard exibe alerta ``Bloqueado'', (4) Gestor recebe notificação de uso não autorizado, (5) Ao publicar 8 saídas, estado volta para ``Liberado''

\vspace{0.3cm}
\textbf{CT-E2E-03: Solicitação e Conclusão de Manutenção}

\textit{Cenário:} Staff solicita manutenção → Equipe aceita → Gestor conclui

\textit{Dados de Entrada:} Solicitação de manutenção para INST-001 com motivo ``Ar condicionado com defeito''

\textit{Resultado Esperado:} (1) Equipe de manutenção recebe e-mail, (2) Estado da instalação = ``Bloqueado (Aguardando manutenção)'', (3) Após conclusão, histórico registra todas as datas/responsáveis, (4) Estado volta para ``Liberado''

\vspace{0.3cm}
\textbf{CT-E2E-04: Solicitação e Conclusão de Limpeza}

\textit{Cenário:} Staff solicita limpeza → Equipe aceita → Gestor conclui

\textit{Dados de Entrada:} Solicitação de limpeza para INST-002

\textit{Resultado Esperado:} (1) Equipe de limpeza recebe notificação, (2) Estado = ``Bloqueado (Aguardando limpeza)'', (3) Histórico completo armazenado, (4) Estado final = ``Liberado''

\vspace{0.3cm}
\textbf{CT-E2E-05: Configuração e Alertas de Limites}

\textit{Cenário:} Gestor altera limite de CO₂ para 800ppm → BIoT recebe novo limite → Sensor envia valor 850ppm

\textit{Dados de Entrada:} Novo limite CO₂=800ppm (via Manager), leitura de sensor CO₂=850ppm

\textit{Resultado Esperado:} (1) BIoT atualiza limite interno em ≤30min, (2) Dashboard exibe alerta ``Limite máximo atingido'', (3) LED de alerta aceso no BIoT, (4) Classificação de risco atualizada no Dashboard

\vspace{0.3cm}
\textbf{CT-E2E-06: Uso Não Autorizado de Instalação}

\textit{Cenário:} Pessoas acessam instalação sem agendamento

\textit{Dados de Entrada:} Evento de entrada (entry\_flow=1) em instalação sem agendamento

\textit{Resultado Esperado:} (1) Manager detecta ocupação não autorizada, (2) Gestor recebe notificação, (3) Evento registrado no log de auditoria

\vspace{0.3cm}
\textbf{CT-E2E-07: Exibição de Histórico no Dashboard}

\textit{Cenário:} Visualizar gráficos temporais da última hora

\textit{Dados de Entrada:} Dados históricos de INST-001 (30 pontos de intervalo de 2min)

\textit{Resultado Esperado:} (1) Dashboard exibe 5 gráficos lineares (temp, CO₂, umidade, VOCs, pessoas), (2) Valores máximos exibidos para cada medida, (3) Intervalo temporal = 2 minutos

\newpage
\subsection{Testes de Requisitos Não Funcionais}

\textbf{Objetivo:} Validar atributos de qualidade do sistema como desempenho, segurança, confiabilidade, usabilidade e manutenibilidade, garantindo que o sistema atende aos requisitos não funcionais especificados.

\textbf{Nota Importante:} Os testes de requisitos não funcionais devem ser executados \textbf{APÓS} a conclusão e aprovação dos testes funcionais, garantindo que a base funcional do sistema está estável antes de avaliar seus atributos de qualidade.

\subsubsection{Testes de Integração e Comunicação (MQTT/Broker)}

\paragraph{Prazo para realização}

\textbf{Momento:} Após aprovação dos testes funcionais\\
\textbf{Duração estimada:} 2 dias\\
\textbf{Frequência:} A cada modificação na estrutura de mensagens MQTT ou atualização de protocolo

\paragraph{Recursos necessários}

\textit{Software:}
\begin{itemize}[noitemsep]
    \item Broker MQTT de teste
    \item Cliente MQTT (MQTT Explorer ou similar)
    \item Scripts de publicação/subscrição MQTT
    \item Analisador de rede (Wireshark) para verificar criptografia
\end{itemize}

\textit{Humanos:}
\begin{itemize}[noitemsep]
    \item 1 testador com conhecimento de MQTT e protocolos de comunicação
\end{itemize}

\paragraph{Requisitos Não Funcionais a serem testados}

\textbf{BIoT:} RNF1, RNF2, RNF10, RNF11, RNF15, RNF17\\
\textbf{Manager:} RNF01, RNF03, RNF07, RNF08, RNF16, RNF17, RNF19

\paragraph{Casos de Teste}

Referência: Documento \textbf{``Casos de Teste SAFE.pdf''} para scripts detalhados.

\textbf{CT-INT-01: Validação de Formato JSON (SAFE\_IAQ)}

\textit{Cenário:} Publicar mensagens MQTT com diferentes formatos JSON (válido, campo faltando, tipo errado, valor NULL, JSON inválido)

\textit{Dados de Entrada:} 5 payloads JSON variando de válido a completamente inválido

\textit{Resultado Esperado:} (1) Payload válido aceito e armazenado, (2) Payloads inválidos rejeitados com alerta ``DADOS DO DISPOSITIVO [ID] EM FORMATO ERRADO''

\vspace{0.3cm}
\textbf{CT-INT-02: Validação de Formato JSON (SAFE\_ENTRY\_FLOW)}

\textit{Cenário:} Testar formato JSON para eventos de entrada/saída

\textit{Dados de Entrada:} entry\_flow=1 (entrada), entry\_flow=-1 (saída), entry\_flow=``entrada'' (inválido)

\textit{Resultado Esperado:} (1) Entrada: contador +1, (2) Saída: contador -1, (3) Inválido: alerta de formato errado

\vspace{0.3cm}
\textbf{CT-INT-03: Criptografia Speck nos Dados}

\textit{Cenário:} Capturar pacotes MQTT com Wireshark durante transmissão de dados

\textit{Dados de Entrada:} BIoT publica dados no tópico SAFE\_IAQ

\textit{Resultado Esperado:} (1) Payload do pacote MQTT criptografado (não legível em texto plano), (2) Manager descriptografa e processa corretamente

\subsubsection{Testes de Limites (Boundary Testing)}

\paragraph{Prazo para realização}

\textbf{Momento:} Após testes de integração e comunicação\\
\textbf{Duração estimada:} 1 dia\\
\textbf{Frequência:} Uma vez por release

\paragraph{Recursos necessários}

\textit{Software:}
\begin{itemize}[noitemsep]
    \item Simulador de tempo (para acelerar/desacelerar intervalos)
    \item Scripts para injetar valores extremos nos sensores
\end{itemize}

\textit{Humanos:}
\begin{itemize}[noitemsep]
    \item 1 testador especializado em testes de borda
\end{itemize}

\paragraph{Requisitos Não Funcionais a serem testados}

Todos os requisitos não funcionais que dependem de \textbf{constantes de tempo e valores limite}.

\textbf{Requisitos específicos:}\\
\textbf{BIoT:} RNF3, RNF4, RNF5, RNF10, RNF12\\
\textbf{Manager:} RNF07, RNF15\\
\textbf{Dashboard:} RNF04, RNF05, RNF06

\paragraph{Casos de Teste}

Referência: Documento \textbf{``Casos de Teste SAFE.pdf''} para scripts detalhados.

\textbf{CT-LIM-01: Timeout de Sessão}

\textit{Cenário:} Usuário autenticado permanece inativo por período superior a TEMPO\_MAXIMO\_INATIVIDADE (1 hora)

\textit{Dados de Entrada:} Login às T0, inatividade por 61 minutos

\textit{Resultado Esperado:} (1) Aos 59min: sessão ainda ativa, (2) Aos 61min: sessão encerrada automaticamente, (3) Redirecionamento para tela de login

\vspace{0.3cm}
\textbf{CT-LIM-02: Intervalo de Atualização de Dados (15s)}

\textit{Cenário:} Monitorar publicações do BIoT no tópico SAFE\_IAQ

\textit{Dados de Entrada:} 10 publicações consecutivas do BIoT

\textit{Resultado Esperado:} Intervalo entre publicações = 15s ± 1s (INTERVALO\_ATUALIZACAO\_DADOS\_QUALIDADE\_AR)

\vspace{0.3cm}
\textbf{CT-LIM-03: Valores Extremos de CO₂}

\textit{Cenário:} Publicar leituras de CO₂ em valores extremos (limite configurado = 1000ppm)

\textit{Dados de Entrada:} CO₂ = 0ppm, 500ppm, 999ppm, 1000ppm, 5000ppm

\textit{Resultado Esperado:} (1) Valores ≤999ppm: aceitos sem alerta, (2) Valores ≥1000ppm: aceitos com alerta, (3) Dashboard exibe valores corretamente

\vspace{0.3cm}
\textbf{CT-LIM-06: Alerta de Falta de Dados (5 minutos)}

\textit{Cenário:} Interromper publicações do BIoT por período superior a TEMPO\_MAXIMO\_ANTES\_ALERTA\_DASHBOARD

\textit{Dados de Entrada:} Última publicação em T0, sem novas publicações

\textit{Resultado Esperado:} (1) T0+4min: sem alerta, (2) T0+5min: Dashboard exibe ``OS DADOS NÃO ESTÃO SENDO ATUALIZADOS'', (3) Ao retomar publicação: alerta desaparece

\subsubsection{Testes de Segurança e Controle de Acesso}

\paragraph{Prazo para realização}

\textbf{Momento:} Após testes de limites\\
\textbf{Duração estimada:} 1 dia\\
\textbf{Frequência:} Uma vez por release e obrigatoriamente após qualquer mudança de segurança

\paragraph{Recursos necessários}

\textit{Software:}
\begin{itemize}[noitemsep]
    \item Múltiplas contas de teste (uma para cada papel de usuário)
    \item Ferramenta de análise de requisições (Postman, Burp Suite)
    \item Scanner de vulnerabilidades (opcional)
\end{itemize}

\textit{Humanos:}
\begin{itemize}[noitemsep]
    \item 1 testador de segurança certificado ou com experiência comprovada
\end{itemize}

\paragraph{Requisitos Não Funcionais a serem testados}

\textbf{Manager:} RNF14, RNF15, RNF16, RNF17\\
\textbf{BIoT:} RNF15, RNF16, RNF17, RNF19

\paragraph{Casos de Teste}

Referência: Documento \textbf{``Casos de Teste SAFE.pdf''} para scripts detalhados.

\textbf{CT-SEG-01: Login com Credenciais}

\textit{Cenário:} Testar autenticação com credenciais válidas e inválidas

\textit{Dados de Entrada:} (1) Email/senha válidos, (2) Senha errada, (3) Email inexistente, (4) Campos vazios

\textit{Resultado Esperado:} (1) Login bem-sucedido, (2-4) Mensagens de erro apropriadas

\vspace{0.3cm}
\textbf{CT-SEG-02: Troca de Senha no Primeiro Acesso}

\textit{Cenário:} Novo usuário faz primeiro login

\textit{Dados de Entrada:} Login com credenciais temporárias, tentativa de nova senha fraca e forte

\textit{Resultado Esperado:} (1) Redirecionamento obrigatório para alteração de senha, (2) Senha fraca rejeitada com critérios exibidos, (3) Senha forte aceita (>6 caracteres, números, maiúsculas, minúsculas, especiais)

\vspace{0.3cm}
\textbf{CT-SEG-04: Controle de Acesso por Papel (RBAC)}

\textit{Cenário:} Usuários de diferentes papéis tentam acessar recursos

\textit{Dados de Entrada:} Login como Administrador, Gestor, Staff, Equipe Manutenção

\textit{Resultado Esperado:} (1) Administrador: acesso total, (2) Gestor: acesso apenas à sua hierarquia, (3) Staff: apenas solicitações, sem configurações, (4) Tentativas de acesso não autorizado: bloqueadas

\subsubsection{Testes de Confiabilidade e Tolerância a Falhas}

\paragraph{Prazo para realização}

\textbf{Momento:} Após testes de segurança\\
\textbf{Duração estimada:} 1 dia\\
\textbf{Frequência:} Uma vez por release

\paragraph{Recursos necessários}

\textit{Software/Hardware:}
\begin{itemize}[noitemsep]
    \item Ferramenta para simular queda de rede
    \item Simulador de queda de energia (para BIoT)
    \item Script para derrubar Broker temporariamente
    \item Ferramenta de monitoramento de uptime
\end{itemize}

\textit{Humanos:}
\begin{itemize}[noitemsep]
    \item 1 testador especializado em testes de confiabilidade
\end{itemize}

\paragraph{Requisitos Não Funcionais a serem testados}

\textbf{BIoT:} RNF6, RNF7, RNF8, RNF9, RNF13\\
\textbf{Manager:} RNF05, RNF06, RNF09, RNF11\\
\textbf{Dashboard:} RNF04, RNF08, RNF10

\paragraph{Casos de Teste}

Referência: Documento \textbf{``Casos de Teste SAFE.pdf''} para scripts detalhados.

\textbf{CT-TOL-01: Queda e Retorno de Internet (BIoT)}

\textit{Cenário:} BIoT perde conexão Wi-Fi e reconecta após 2 minutos

\textit{Dados de Entrada:} Desabilitar Wi-Fi, aguardar 2min, reabilitar Wi-Fi

\textit{Resultado Esperado:} (1) Durante offline: BIoT armazena dados localmente, (2) Reconexão automática em <30s, (3) Publicação de backlog + dados novos, (4) Manager recebe todos os dados

\vspace{0.3cm}
\textbf{CT-TOL-06: Dados em Formato Errado}

\textit{Cenário:} Publicar JSON malformado no Broker

\textit{Dados de Entrada:} \texttt{\{temperatura: vinte, co2: abc\}} (JSON inválido)

\textit{Resultado Esperado:} (1) Manager detecta formato inválido, (2) Alerta na tela: ``DADOS DO DISPOSITIVO [ID] EM FORMATO ERRADO'', (3) Erro registrado em log com timestamp

\vspace{0.3cm}
\textbf{CT-TOL-07: Disponibilidade 99,9\%}

\textit{Cenário:} Monitoramento contínuo do Manager e Dashboard por 30 dias

\textit{Dados de Entrada:} Ferramenta de monitoramento de uptime

\textit{Resultado Esperado:} Disponibilidade ≥99,9\% ao mês (downtime máximo ≈43 minutos), exceto janelas de manutenção programadas

\subsubsection{Testes de Desempenho e Eficiência}

\paragraph{Prazo para realização}

\textbf{Momento:} Após testes de confiabilidade\\
\textbf{Duração estimada:} 1 dia\\
\textbf{Frequência:} Uma vez por release e obrigatoriamente antes do deploy em produção

\paragraph{Recursos necessários}

\textit{Software/Hardware:}
\begin{itemize}[noitemsep]
    \item Gerador de carga (JMeter, Locust ou script customizado)
    \item Monitor de recursos do sistema (CPU, Memória, Rede, Disco)
    \item Equipamento de referência para teste de precisão de sensores
\end{itemize}

\textit{Humanos:}
\begin{itemize}[noitemsep]
    \item 1 testador de performance com experiência em sistemas IoT
\end{itemize}

\paragraph{Requisitos Não Funcionais a serem testados}

\textbf{BIoT:} RNF5 (Precisão 90\%), RNF10, RNF13\\
\textbf{Manager:} RNF07, RNF09 (Disponibilidade 99,9\%)\\
\textbf{Dashboard:} RNF05, RNF07

\paragraph{Casos de Teste}

Referência: Documento \textbf{``Casos de Teste SAFE.pdf''} para scripts detalhados.

\textbf{CT-PERF-01: Carga com 50 Dispositivos BIoT Simultâneos}

\textit{Cenário:} Simular 50 dispositivos BIoT publicando dados simultaneamente a cada 15s

\textit{Dados de Entrada:} Gerador de carga publicando 50 streams MQTT paralelos

\textit{Resultado Esperado:} (1) Manager processa todos os dados sem perda, (2) Latência de processamento <5s, (3) CPU/Memória do Manager dentro de limites aceitáveis

\vspace{0.3cm}
\textbf{CT-PERF-02: Precisão de Leitura dos Sensores (≥90\%)}

\textit{Cenário:} Comparar 100 leituras do BIoT com equipamento de referência

\textit{Dados de Entrada:} 100 leituras simultâneas de temperatura, CO₂, umidade e 50 eventos de contagem de pessoas

\textit{Resultado Esperado:} Erro médio ≤10\% para cada medida (precisão ≥90\% conforme RNF5)

\vspace{0.3cm}
\textbf{CT-PERF-03: Latência de Atualização Dashboard (≤15s)}

\textit{Cenário:} Medir tempo entre publicação BIoT e exibição no Dashboard

\textit{Dados de Entrada:} Publicação de dados no Broker com timestamp

\textit{Resultado Esperado:} Dashboard atualiza valores em ≤15s após publicação (INTERVALO\_ATUALIZAR\_DADOS)

\subsubsection{Testes de Usabilidade (Interface Responsiva e Compatibilidade)}

\paragraph{Prazo para realização}

\textbf{Momento:} Após testes de desempenho\\
\textbf{Duração estimada:} 0.5 dia\\
\textbf{Frequência:} Uma vez por release

\paragraph{Recursos necessários}

\textit{Software/Hardware:}
\begin{itemize}[noitemsep]
    \item Dispositivos físicos: Smartphone, Tablet, Desktop
    \item Ferramenta de teste responsivo (Browser DevTools)
    \item Navegadores: Chrome 118+, Firefox 115+, Edge 118+, Safari 16+
\end{itemize}

\textit{Humanos:}
\begin{itemize}[noitemsep]
    \item 1 testador de UI/UX com conhecimento em design responsivo
\end{itemize}

\paragraph{Requisitos Não Funcionais a serem testados}

\textbf{Manager:} RNF13, RNF18\\
\textbf{Dashboard:} RNF12, RNF13, RNF14

\paragraph{Casos de Teste}

Referência: Documento \textbf{``Casos de Teste SAFE.pdf''} para scripts detalhados.

\textbf{CT-RESP-01: Layout Pequeno (≤640px)}

\textit{Cenário:} Acessar Dashboard em dispositivo móvel (360x640px)

\textit{Dados de Entrada:} DevTools configurado para resolução mobile

\textit{Resultado Esperado:} (1) Cards de instalações empilhados verticalmente, (2) Menu responsivo (hamburguer), (3) Todos os elementos acessíveis e funcionais

\vspace{0.3cm}
\textbf{CT-RESP-04: Compatibilidade de Navegadores}

\textit{Cenário:} Acessar Manager e Dashboard em diferentes navegadores

\textit{Dados de Entrada:} Chrome 118+, Firefox 115+, Edge 118+, Safari 16+

\textit{Resultado Esperado:} (1) Dashboard carrega corretamente em todos, (2) Gráficos exibidos, (3) Alertas funcionam, (4) Login do Manager funcional

\subsubsection{Testes de Manutenibilidade (Instalação e Atualização)}

\paragraph{Prazo para realização}

\textbf{Momento:} Após testes de usabilidade, antes de cada deploy\\
\textbf{Duração estimada:} 0.5 dia\\
\textbf{Frequência:} Obrigatoriamente a cada atualização de versão

\paragraph{Recursos necessários}

\textit{Software/Hardware:}
\begin{itemize}[noitemsep]
    \item Ambiente limpo (máquina virtual ou container sem instalação prévia)
    \item Docker instalado
    \item Acesso ao repositório de imagens Docker
\end{itemize}

\textit{Humanos:}
\begin{itemize}[noitemsep]
    \item 1 testador de infraestrutura ou DevOps
\end{itemize}

\paragraph{Requisitos Não Funcionais a serem testados}

\textbf{Manager:} RNF10, RNF11, RNF12\\
\textbf{Dashboard:} RNF09, RNF10, RNF11\\
\textbf{BIoT:} RNF14, RNF20, RNF21

\paragraph{Casos de Teste}

\textbf{CT-INST-01: Instalação Inicial via Docker (Manager)}

\textit{Cenário:} Instalar Manager em ambiente limpo usando Docker

\textit{Dados de Entrada:} Imagem Docker do Manager, ambiente sem instalação prévia

\textit{Resultado Esperado:} (1) Container inicia sem erros, (2) Banco de dados criado automaticamente, (3) Manager acessível via browser, (4) Logs sem erros críticos

\vspace{0.3cm}
\textbf{CT-INST-03: Atualização de Versão sem Perda de Dados (Rollout)}

\textit{Cenário:} Atualizar Manager da v1.0 para v1.1 com dados existentes

\textit{Dados de Entrada:} Sistema v1.0 com dados de produção, nova imagem Docker v1.1

\textit{Resultado Esperado:} (1) Atualização bem-sucedida, (2) Sem perda de dados do BD, (3) Sistema funcional após atualização, (4) Possibilidade de rollback

\vspace{0.3cm}
\textbf{CT-INST-04: Acesso Direto ao Servidor em Caso de Falha Sistêmica}

\textit{Cenário:} Sistema com falha sistêmica, Administrador precisa acessar servidor

\textit{Dados de Entrada:} Credenciais de administrador, acesso SSH/terminal

\textit{Resultado Esperado:} (1) Administrador consegue acesso direto ao servidor, (2) Logs acessíveis, (3) Comandos de diagnóstico funcionais

\newpage
\section{Resultados dos Testes}

Esta seção será preenchida após a execução dos testes. Para cada tipo de teste, registrar:

\begin{itemize}
    \item Data de execução
    \item Testador responsável
    \item Ambiente utilizado
    \item Casos de teste executados
    \item Taxa de sucesso/falha
    \item Defeitos encontrados (IDs de rastreamento)
    \item Ações corretivas tomadas
    \item Observações e recomendações
\end{itemize}

\subsection{Resultados - Testes Funcionais E2E}

\begin{tabularx}{\textwidth}{|l|X|l|l|l|l|X|}
\hline
\rowcolor{gray!20} \textbf{Data} & \textbf{Testador} & \textbf{Ambiente} & \textbf{Casos Exec.} & \textbf{Sucesso} & \textbf{Falha} & \textbf{Observações} \\ \hline
& & & & & & \\ \hline
\end{tabularx}

\vspace{0.5cm}
\textbf{Defeitos Encontrados:}
\begin{itemize}
    \item [Lista de bugs com IDs e severidade]
\end{itemize}

\textbf{Ações Tomadas:}
\begin{itemize}
    \item [Descrição das correções aplicadas]
\end{itemize}

\subsection{Resultados - Testes de Integração MQTT}

\subsection{Resultados - Testes de Limites}

\subsection{Resultados - Testes de Segurança}

\subsection{Resultados - Testes de Tolerância a Falhas}

\subsection{Resultados - Testes de Desempenho}

\subsection{Resultados - Testes de Responsividade}

\subsection{Resultados - Testes de Instalação}

\end{document}